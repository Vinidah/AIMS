%%%%%%%%%%%%%%%%%%%%%%%%%%%%%%%%%%%%%%%%%%%%%%%%%%%%%%%%%%%%%%%%%%%%%%%%%%%%%%%%
%%%%%%%%%%%%%%%%%%%%%%%%%%%%%%%%%%%%%%%%%%%%%%%%%%%%%%%%%%%%%%%%%%%%%%%%%%%%%%%%
%%% Template for AIMS Rwanda Assignments         %%%              %%%
%%% Author:   AIMS Rwanda tutors                             %%%   ###        %%%
%%% Email: tutors2017-18@aims.ac.rw                               %%%   ###        %%%
%%% Copyright: This template was designed to be used for    %%% #######      %%%
%%% the assignments at AIMS Rwanda during the academic year %%%   ###        %%%
%%% 2017-2018.                                              %%%   #########  %%%
%%% You are free to alter any part of this document for     %%%   ###   ###  %%%
%%% yourself and for distribution.                          %%%   ###   ###  %%%
%%%                                                         %%%              %%%
%%%%%%%%%%%%%%%%%%%%%%%%%%%%%%%%%%%%%%%%%%%%%%%%%%%%%%%%%%%%%%%%%%%%%%%%%%%%%%%%
%%%%%%%%%%%%%%%%%%%%%%%%%%%%%%%%%%%%%%%%%%%%%%%%%%%%%%%%%%%%%%%%%%%%%%%%%%%%%%%%


%%%%%% Ensure that you do not write the questions before each of the solutions because it is not necessary. %%%%%% 

\documentclass[12pt,a4paper]{article}

%%%%%%%%%%%%%%%%%%%%%%%%% packages %%%%%%%%%%%%%%%%%%%%%%%%
\usepackage{amsmath}
\usepackage{amssymb}
\usepackage{amsthm}
\usepackage{amsfonts}
\usepackage{graphicx}
\usepackage{svg}
\usepackage{float}
\usepackage[all]{xy}
\usepackage{tikz}
\usepackage{verbatim}
\usepackage[left=2cm,right=2cm,top=3cm,bottom=2.5cm]{geometry}
\usepackage{hyperref}
\usepackage{caption}
\usepackage{subcaption}
\usepackage{psfrag}
\usepackage{placeins}
%%%%%%%%%%%%%%%%%%%%% students data %%%%%%%%%%%%%%%%%%%%%%%%
\newcommand{\student}{Yusuf Brima}
\newcommand{\course}{Operations Research}
\newcommand{\assignment}{1}

%%%%%%%%%%%%%%%%%%% using theorem style %%%%%%%%%%%%%%%%%%%%
\newtheorem{thm}{Theorem}
\newtheorem{lem}[thm]{Lemma}
\newtheorem{defn}[thm]{Definition}
\newtheorem{exa}[thm]{Example}
\newtheorem{rem}[thm]{Remark}
\newtheorem{coro}[thm]{Corollary}
\newtheorem{quest}{Question}[section]

%%%%%%%%%%%%%%  Shortcut for usual set of numbers  %%%%%%%%%%%

\newcommand{\N}{\mathbb{N}}
\newcommand{\Z}{\mathbb{Z}}
\newcommand{\Q}{\mathbb{Q}}
\newcommand{\R}{\mathbb{R}}
\newcommand{\C}{\mathbb{C}}

%%%%%%%%%%%%%%%%%%%%%%%%%%%%%%%%%%%%%%%%%%%%%%%%%%%%%%%555
\begin{document}

%%%%%%%%%%%%%%%%%%%%%%% title page %%%%%%%%%%%%%%%%%%%%%%%%%%
\thispagestyle{empty}
%\begin{figure}
%    \centering
%    \includegraphics[width=\textwidth]{aims_rwanda.jpg}
%\end{figure}
\begin{center}
\textbf{AFRICAN INSTITUTE FOR MATHEMATICAL SCIENCES \\[0.5cm]
(AIMS RWANDA, KIGALI)}
\vspace{1.0cm}
\end{center}

%%%%%%%%%%%%%%%%%%%%% assignment information %%%%%%%%%%%%%%%%
\noindent
\rule{17cm}{0.2cm}\\[0.3cm]
Name: \student \hfill Assignment Number: \assignment\\[0.1cm]
Course: \course \hfill Date: \today\\
\rule{17cm}{0.05cm}
\vspace{1.0cm}
\section*{Question 2}
Consider the problem
\begin{equation}
    \begin{aligned}
    \text{Minimize } \quad    z &= 3x_1 - x_2 + x_3 \\
    \text{subject to }\quad &
        \begin{array}{c}
            x_1 +2 x_2 \leq 4 \\
           2 x_1 - x_2 + x_3 \geq 1 \\
            x_1,x_2 \geq 0, x_3 \leq 0
        \end{array}
    \end{aligned}
    \label{eq:problem_2}
\end{equation}
\begin{enumerate}
	\item[(a)] Is the point $(x_1, x_2, x_3) = (\frac{6}{5}, \frac{7}{5}, 0)$ an optimal solution?
	\item[(b)] Is the point $(x_1, x_2, x_3) = (\frac{1}{2}, 0, 0)$ an optimal solution?
\end{enumerate}
\section*{Question 3}
Study the following LP problem using duality concept.
\begin{equation}
    \begin{aligned}
    \text{Maximize } \quad    z &= z = c_1x_1 + c_2x_2 \\
    \text{subject to }\quad &
        \begin{array}{c}
            –x_1 + x_2  \leq 1 \\
            x_1 + 2x_2  \leq 2 \\
           2x_1 + x_2 \geq 0 \\
           2x_1 – 2x_2  \leq 1 \\
            x_1,x_2 \geq 0
        \end{array}
    \end{aligned}
    \label{eq:problem_3}
\end{equation}
\begin{enumerate}
	\item[(a)] Determine the primal solution where the dual variables are given by the vector\\
		$v = (0, 1/3, 0, 2/3)^\text{T}$\\
		By using Complementary Slackness \eqref{eq_comp_slack}
		\begin{equation}
			\bar{W}^\text{T}(B - A\bar{X}) = 0 \quad  \text{or} \quad  \bar{X}^\text{T}(A^\text{T}\bar{W} - C) = 0
			\label{eq_comp_slack}
		\end{equation}
		Therefore, $\bar{W}^\text{T}(B - A\bar{X}) = 0 $ is solved as follows:
		\begin{align*}
				0(1 - (-x_1 + x_2)) &= 0\\
				\frac{1}{3} (2  - (x_1 + 2x_2) &= 0\\
				0(0 - (2x_1 + x_2)) &= 0\\
				\frac{2}{3} (1 - (2x_1 - 2x_2)) &= 0\\
		\end{align*}
		We therefore,  get the following system of simultaneous equations as stated below.
		\begin{align*}
				x_1 + 2x_2  &= 2 \\
				 2x_1 - 2x_2 &= 1
				 \label{eq:pset_3_sim}
		\end{align*}
		Solving the above simultaneous equation where $x_1 = 1$ and $x_2 = \frac{1}{2}$; the solution for the primal is thus:
			$\bar{X}^\text{T} = (1,\frac{1}{2})^\text{T}$
	\item[(b)] Determine the values of $c_1$ and $c_2$ for which the primal solution is optimal.\\
		New $ \bar{X}^\text{T}(A^\text{T}\bar{W} - C) = 0$\\
		To solve for $c_1$ and $c_2$, we first determine the dual to the Linear Programming Problem in \eqref{eq:problem_3} as follows:
\begin{equation}
    \begin{aligned}
    \text{Minimize }  \quad  v &=  w_1 + 2w_2 + 0w_3 + w4 \\
    \text{subject to }\quad &
        \begin{array}{c}
        	   -w_1 + w_2 + 2w_3 + 2w_4 \geq c_1 \\
        	   w_1 + 2w_2 + w_3 - 2w_4 \geq c_2 \\
        	   w_1,w_2 \geq 0, w_3 \leq 0, w_4 \geq 0
        \end{array}
    \end{aligned}
    \label{eq:problem_3_b}
\end{equation}
$\bar{X}^\text{T}(A^\text{T}W - C) = 0 $\\
From \eqref{eq:problem_3_b}, where therefore solve for $c_1$ and $c_2$ as follows:
\begin{align*}
		\text{(1)} 1(-w_1 + w_2 + 3w_3 + 2w_4 - c_1) &=0\\
		(0 + \frac{1}{3} + 0 + \frac{4}{3}  - c_1) &= 0\\
		c_1 &= \frac{5}{3}
\end{align*}

\begin{align*}
		\text{(2)} \frac{1}{2}(w_1 + 2w_2 + w_3 - 2w_4 - c_2) &= 0\\
					(0 + \frac{2}{3} + 0  - \frac{4}{3} - c_2) &= 0\\
					c_2 &= \frac{-2}{3}
\end{align*}
\end{enumerate}
\end{document}